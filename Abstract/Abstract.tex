\documentclass[conference]{IEEEtran}
\usepackage[utf8]{inputenc}
\usepackage{graphicx}
\usepackage{amsmath}
\usepackage{cite}
\usepackage{url}
\usepackage[none]{hyphenat}
\usepackage{float}
\usepackage{tikz}
\usepackage{algorithm}
\usepackage{algpseudocode}
\usepackage{microtype}
\usepackage{ragged2e}
\usepackage[colorlinks=true,citecolor=black,linkcolor=black,urlcolor=black]{hyperref}

\tolerance=1%
\emergencystretch=\maxdimen%
\hyphenpenalty=10000%
\exhyphenpenalty=100%

\usetikzlibrary{arrows.meta, positioning, shapes.geometric}

\title{Anti-Plagiarism System for Exam Monitoring}

\author{
    \IEEEauthorblockN{Valentin Pletea-Marinescu}
    \IEEEauthorblockA{
        \textit{National University of Science and Technology POLITEHNICA Bucharest}\\
        Email: \texttt{pletea.valentin2003@gmail.com}
    }
}

\begin{document}

\maketitle

\begin{abstract}
Academic integrity represents a fundamental challenge in modern education systems, 
with plagiarism rates increasing globally across all educational levels. Traditional 
exam monitoring approaches rely primarily on screen surveillance and human oversight, 
creating significant vulnerabilities in detecting sophisticated cheating behaviors 
during online and remote examinations.
This research develops a comprehensive anti-plagiarism monitoring system using 
deep learning and computer vision technologies to address these limitations. 
The system integrates gaze tracking algorithms with YOLO-based object detection 
models through a modular software architecture. Facial landmark detection enables 
precise gaze direction analysis, while specialized convolutional neural networks 
identify unauthorized objects and suspicious materials in the examination environment. 
These two complementary approaches work together to provide comprehensive monitoring 
coverage: gaze analysis detects abnormal visual attention patterns that may indicate 
unauthorized assistance seeking, while object detection identifies physical cheating 
aids such as smartphones and smartwatches.
Experimental testing demonstrates effective operation on CPU-based 
systems, identifying suspicious gaze patterns, abnormal behaviors, 
and unauthorized objects while maintaining real-time performance. The system provides 
comprehensive exam monitoring capabilities suitable for educational institutions, enabling 
widespread deployment using standard computing hardware without requiring specialized equipment.
\end{abstract}

\begin{IEEEkeywords}
Academic integrity, Educational technology, Computer vision, Gaze tracking, Object detection, Real-time systems, Machine learning, Convolutional neural networks, Image processing, Kalman Filters
\end{IEEEkeywords}

\end{document}