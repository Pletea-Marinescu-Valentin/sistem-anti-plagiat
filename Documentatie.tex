\documentclass[a4paper,12pt]{article}

\usepackage[utf8]{inputenc}
\usepackage[romanian]{babel}
\usepackage{hyperref}

\title{Documentație Anti-Plagiat}
\author{Valentin}
\date{\today}

\begin{document}
\hypersetup{
    colorlinks=true,
    linkcolor=blue,
    filecolor=magenta,
    urlcolor=cyan,
    pdftitle={Documentație Anti-Plagiat},
    pdfpagemode=FullScreen,
}
\maketitle

\tableofcontents
\newpage

\section{Introducere}
Acest proiect, intitulat \textbf{Anti-Plagiat}, are ca scop dezvoltarea unui sistem care să detecteze și să prevină plagiatul în documente. Documentația de față oferă o descriere generală a proiectului și a funcționalităților sale.

\section{Structura Proiectului}
Proiectul este organizat în mai multe componente, fiecare având un rol specific:
\begin{itemize}
    \item \textbf{Modul de analiză}: Analizează documentele pentru similarități.
    \item \textbf{Interfața utilizator}: Permite utilizatorilor să încarce documente și să vizualizeze rezultatele.
    \item \textbf{Baza de date}: Stochează documentele și rezultatele analizelor.
\end{itemize}

\section{Tehnologii Utilizate}
Proiectul utilizează următoarele tehnologii:
\begin{itemize}
    \item \textbf{Limbaj de programare}: Python
    \item \textbf{Biblioteci}: NumPy, Pandas, Scikit-learn
    \item \textbf{Alte tehnologii}: LaTeX pentru documentație
\end{itemize}

\section{Concluzie}
Proiectul \textbf{Anti-Plagiat} reprezintă un pas important în prevenirea plagiatului, oferind un instrument eficient și ușor de utilizat. Documentația va fi extinsă pe măsură ce proiectul evoluează.

\end{document}